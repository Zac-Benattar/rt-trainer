\chapter{Introduction}
\label{ch:introduction}

Write around four paragraphs establishing the context and motivating your project.

Radiotelephony (R/T) is the language of flight crew and air traffic control (ATC) in radio communication. As a specifically structured subset of spoken English, it provides a global standard for aircraft communication when supplemented with Aviation English. Though many countries with high aircraft traffic have their own versions, they all closely follow the international standard as defined by the International Civil Aviation Organisation (ICAO) in Document 9432 \cite{Doc9432}. In the UK, prospective pilots and flight crew must pass both a R/T theory and practical exam as part of their licensing process. A common process for practising for the practical exam involves learning the CAP413 document which defines standard R/T in UK airspace \cite{CAP413}, then organising mock tests with an examiner. Organising mock tests is expensive and time consuming, hence practice software that imitates exam scenarios are frequently used as part of a student's practice routine. The main issue with existing practice software is the lack of speech input capability. Requiring users to enter their radio messages in text form somewhat limits the quality of the practice. It would thus be useful for trainee pilots to have access to a solo practice system which provides mock test like conditions without the need for an examiner to play the part of ATC, while supporting speech input.

\section{Related work}

Although limited by a lack of voice input support, two high quality and popular R/T practice applications are currently available online.

Readability5 presents a simple interface which mimics the resources a student would have in the exam \cite{Readability5}, that being a radio, transponder, map, and kneeboard (a notepad often attached to a pilot's upper thigh). In each module the user is walked through a scenario in which, mimicking the actual exam, they must select the correct frequencies and modes on both the radio and transponder and transmit radio calls. Five modules which teach different aspects of R/T can be purchased, however the route used in each module is fixed, so once a module has been completed once, the next practice will be on the exact same route. Voice input is technically supported, but all speech is considered correct, so users can simply press the transmit button without saying anything and progress to the next state in the scenario.

Wilco Radio is a feature rich training system which provides many practice modes including mock tests \cite{Wilco-Radio}. Its interface shows much more information than that of Readability5, much of it being unnecessary for the exam. It also does not support the managing of radio and transponder frequencies and modes, which are part of the skills required to pass the R/T exam. Its fundamental flaw is the form of input it supports - users select from a set of predefined options given by the system to respond to each radio message. Despite this, it does provide extensive feedback on radio calls, and shows the user's progress metrics including the number of procedures practised, and the user's accuracy rate.
% Add images

\section{Objectives}

RT-Trainer, as the system has been named, aims to recreate many of the 

\begin{enumerate}
    \item Generation of random scenarios with suitable random locations, events, and radio messages for R/T exam practice.
    \item Allow users to practice the scenarios by typing or speaking each radio call when prompted.
    \item Support for up-to-date modern browsers (at least Chrome, Safari, Firefox and Edge).
    \item Feedback on the accuracy of a user's radio calls.
    \item User statistics and progress tracking.
    \item Allow sections of scenario to be practised separately.
    \item Provide multiple levels of support while practising.
    \item Allow sharing of scenarios between users.
\end{enumerate}

Note that objective 7 was not met so is not included in the implementation section. Objective 8 was completed before 7 due to the design of the system providing the means for users to share scenarios without any extra code.
